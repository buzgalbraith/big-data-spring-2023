\documentclass{article}
\usepackage[utf8]{inputenc}
\usepackage{tikz,graphicx,hyperref,amsmath,amsfonts,amscd,amssymb,bm,cite,epsfig,epsf,url}

\title{Dask: Parallel Computation with Blocked algorithms
and Task Scheduling}
\author{wbg231 }
\date{January 2023}

\begin{document}

\maketitle

\section{abstract }
\begin{itemize}
\item dask enables parallel and out of core computation. 
\item we couple vlocked algorirhsm wirh memory awate task scheudling to achive a parallel out of core Numpy clone 
\item this scales to modern hardware and large datasets 
\section*{Introduction}
\item sci py stuff does not use parallel implementations 
\item we want to parallelize the scipy code with out needing a full re-write
\item dask encodes parallel algorithms using python primitives, and has hte dask.array type a parallel n dimensional array that copies numpy's interface
\subsection*{modern hardware}
\item hardware has changed a lot in recent years 
\item most modern cpu's have multiple threads, most modern storage is on an ssd which makes reading information from disk much faster and thus more practice
\item these advancements make single machine implementations rival small cluster computation while keeping the ease of working with a single machine 
\subsection*{dask graphs}
\item dask encodes parallel computation in a way that requires low amount of instruction by the developer
\item a dask graph is a python dictionary mapping keys or tasks to values 
\item storing programs in graphs allows for easy task scheduling 
\subsection*{specification }
\item represent computation as a DAG of tasks with data dependincies 
\item a task is a tuple with  a callable first ellement 
\item tasks are automic unfits of work that can be run by a single worker
\item an argument may be either a key present in the dask, a literal, another task,, or a list of arguments
\subsection*{dask arrays}
\item 
\end{itemize}
\end{document}
