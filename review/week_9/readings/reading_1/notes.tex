\documentclass{article}
\usepackage[utf8]{inputenc}
\usepackage{tikz,graphicx,hyperref,amsmath,amsfonts,amscd,amssymb,bm,cite,epsfig,epsf,url}

\title{reproducibility reading materials }
\author{wbg231 }
\date{January 2023}

\begin{document}

\maketitle

\section{reproducibility handout}
\begin{itemize}
\item using data management can help reproducibility 
\item keep a read me 
\item keep back ups of important files 
\item keep documentation in both scripts and data
\item should not have more than a few sub-folders 
\item file should be named descriptively
\item use a file format that is easy for other to work with 
\item run your scripts top to bottom 
\item make your scripts use relative instead of direct paths 
\item comment scripts
\item use git to have version history
\item decare all dependencies at the top of the scripts
\item name scripts in a logical order 
\item document the dependencies and computational environment fo your scripts
\section*{reproducibility coding}
\item as science has progressed the level of computational skill required has risen in such a way that it is difficult for scientists to keep up
\item researchers need to commend code and make it more open for reproducibility
\item reaserchers who are untrained or do not comment results can bias sicentific results
\subsection*{reproducibility cloture}
\item edditors and authors are often unwilling or unable to address issues in sicentific results
\item 
\end{itemize}
\end{document}
