\documentclass{article}
\usepackage[utf8]{inputenc}
\usepackage{tikz,graphicx,hyperref,amsmath,amsfonts,amscd,amssymb,bm,cite,epsfig,epsf,url}

\title{reproducibility lecture }
\author{wbg231 }
\date{January 2023}

\begin{document}

\maketitle

\section{Introduction}
\begin{itemize}
\item explains her role 
\item how to convert from one file format to another 
\item can make an appointment about many topics
\item reproducibility has been a consistent concern for a long time 
\item reproducing research is hard 
\item the majority of studies across all dispense were found to be irreducible  
\item these can have effects that are important
\item reproducibility is when independent people use the same code and data to verify results
\item code and data reproducibility are part of this, we want to make the entire pipeline reproducible 
\item replication, independent people use the same data and dirent analysis to validate results
\item reviewable research is just the article 
\item open and reproducible research means we have the all the data and the computational frame work 
\item we need to be able to test the reproducibility of results
\item even software results can vary depending on the hardware or operating system that is being used
\item reproducibility allows one to use current work or code more easily down the road
\subsection*{methods }
\item need to have set practices for data and code management
\item at the start of project set the following 
\begin{enumerate}
    \item storage adn backup 
    \item file formats 
    \item project structure 
    \item version control 
    \item documentation 
    \item and group roles
\end{enumerate}
\item following these tools will make sure the work is understandable by machine but humans as well 
\item should have 3 copies of all research material one in a different physical location than you 
\item original copy, cloud service and external hard drive 
\item the life time of a external hard drive is like 3 to 5 years 
\item nyu google drive good for moderate use data 
\item box is good for big data with security concerns 
\item nyu hpc can also let you back up and store data 
\item if working with highly sensitive data ask questions 
\subsection*{project structure and documentation}
\item it is easy to louse your research data 
\item do not put documentation in file path put it at a readme file 
\item make sure to explain the version of your pyhton 
\item also comment how the data is collected and interpreted might help
\item if using jupyter can write what you are doing while you are writing it
\item file naming is important keep choices logical 
\item add prefixes to you data with the data created YYYY-MM-DD so you can sort it easily
\item don't use spaces in your file names 
\item keep file names short, a file name that is   
\item keep a standard way of organizing projects 
\item try keeping each project in its own directory
\item put documentation in docs folder 
\item put data in a data folder 
\item code in a src folder 
\item results in a results folder 
\subsection*{file types}
\item use a common file type that is file agnostic 
\subsection*{version control}
\item try to use version control if you are writing code 
\item github enables collaboration can undo and re-do version control 
\item git is a free open source tool that can be used to store gitrepos 
\item git works best with plane text formats 
\item use git to keep track over versions 
\item once we have established habits for code and code environments 
\subsection*{computational environments}
\item there are tools that will help with computational reproducibility
\item containers are pretty common and liked 
\item web based ides are pretty popular as well 
\item when processing and analyzing tools dockers can be good 
\item web based are good for longer term reproducibility
\subsection*{using containers}
\item a way to produce virtual operating system outside of the operating system 
\item docker is the most popular 
\item learning how to make containers work can be tough 
\subsection*{web based ides}
\item web based ides are a nice 
\item just ide's on the browser, allow users to export or share work in the web 
\item whole tale is popular for reproducibility
\item web ides require that you work in there platform first. that can be tough if you need to different computing environments
\subsection*{web based replay system}
\item take links to code hosted elsewhere and makes a runnable ide for git 
\item this is pretty flexible 
\item \textbf{binder} is pretty popular for this 
\item packaging tools export all of the computational environments you used 
\item they pack everything you used in one place
\item reprozip traces all steps and dependencies and can make a protbale bundle for everything that needs to be re ran later
\item try things out and see what works well. 
\end{itemize}
\end{document}
