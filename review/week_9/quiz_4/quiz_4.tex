\documentclass{article}
\usepackage[utf8]{inputenc}
\usepackage{tikz,graphicx,hyperref,amsmath,amsfonts,amscd,amssymb,bm,cite,epsfig,epsf,url}

\title{title }
\author{wbg231 }
\date{January 2023}

\begin{document}

\maketitle

\section{Introduction}
\begin{itemize}
\item question 1:Locality-sensitive hashing makes MinHash faster by distributing the computation to be located on the same machine as the data being processed.
\item that is false, local sensitive hashing has nothing to do with distributed computation
\item MinHash signatures are generated by applying multiple independent hash functions to the elements of a set, and choosing the minimum value produced by all hash functions.
\item false, you pick the min value for each element
\item 
Adding a new permutation to an existing MinHash signature table can result in a smaller candidate set.
\item false: 
\item How does the estimation of Jaccard similarity change when you use imperfect hashes instead of permutations?  Do you expect the estimated similarity to be higher, lower, or the same?  Why?
\item if we are using imperfect hashes we will have more collisions so we will estimate the Jaccard similarity to be higher than we would using permutations.
\item version control is part of reproducibility
\item true 
\item nyu library recommends we use git for version control
\item true
\item nyu recommends sensitive data is stored on 
\item box
\item hat do NYU research librarians recommend where to put contextual information about the analysis and data files?
\item in a readme at the front of the directory
\item 
To maximize reproducibility, a project folder should contain the following subfolders (select all that apply) 
\item data, src, results, docs, readme
\end{itemize}
\end{document}
