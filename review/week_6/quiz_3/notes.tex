\documentclass{article}
\usepackage[utf8]{inputenc}
\usepackage{tikz,graphicx,hyperref,amsmath,amsfonts,amscd,amssymb,bm,cite,epsfig,epsf,url}

\title{title }
\author{wbg231 }
\date{January 2023}

\begin{document}

\maketitle

\section{Introduction}
\begin{itemize}
\subsection*{q1}
\item \textcolor{blue}{If no caching is involved, wide dependencies are just as fast as narrow dependencies }
\item this if false, regardless of if we are caching (ie) saving data to memory a wide dependencie has to use data from multiple paritions so it will have latency 
\subsection*{question 3}
\item \textcolor{blue}{An RDD can depend on multiple parent and be reused by multiple descendents}
\item true 
\item \textcolor{blue}{The Dremel system was designed to efficiently process subsets of attributes over all records in a dataset.}
\item yep that is the point to look at cols not rows
\item \textcolor{red}{If a numerical column A has been compressed with run-length encoding, it must be decompressed to compute the average mean(A).}
\item run length encoding maps a vector to each of its unique values and the number of times they appeard so false 
\item descibe the roles of paritions in spark RDD's what do they do how do they effect disrivuted computing
\item paritions are the amount of data that a worker node in RDD needs to work with 
\item we want the data in that parition to be as close as possible to the block the worker node is on 
\item raising the number of paritions beyond the number of workers means that after finishing one run, a worker will have to go back and do another (this increses comuncaiton costs) 
\item lowering the value of partions bellow the number of workers means that some worers are dealing with a lot of data while others are idle
\item 
\end{itemize}
\end{document}
