\documentclass{article}
\usepackage[utf8]{inputenc}
\usepackage{tikz,graphicx,hyperref,amsmath,amsfonts,amscd,amssymb,bm,cite,epsfig,epsf,url}

\title{Dremel: Interactive Analysis of Web-Scale Datasets }
\author{wbg231 }
\date{January 2023}

\begin{document}

\maketitle

\section{abstract}
\begin{itemize}
\item dremel is a scalable interactive, ad-hco query system for analysis of read only nested data 
\item is fast as it combines multi level ecxeution trees with columnar data layout
\item this scales both with data and cpus
\subsection*{introduction }
\item data used in science and the internet is often non relational 
\item hence a flexible data model is essential in these domains 
\item a lot of this data lends it's selt naturally to nested representations 
\item dremel is a system that supports interactive analysis of very large data sets over shared clusters of commodity machiens 
\item dremel works well with nested data and works well with map reduce
\item the architecture of dremal uses serving tree structure used in web search 
\item the result of the auery is asembled by aggregatigng the replies recived from lower level of trees 
\item dremel proied a high level sql like language to express ad hoc queries 
\item it can execute queries with out map reduce jobs 
\item dremel is also column oriented which enables it to read less data from seconday storacges adn have chpeater compressin
\subsection*{background }
\item there needs to be interoperation between the query processor and ohter data manamgnet tools
\item ie they need to be able to work togther as a combined system
\item this reuqires a common storage later 
\item the next step of this is a shared storage format, coloumnar storage is good for relational data but dremel helps with nested data 
\subsection*{data model}
\item i am going to go over the lecture and come back to this if there is time. 
\end{itemize}
\end{document}
