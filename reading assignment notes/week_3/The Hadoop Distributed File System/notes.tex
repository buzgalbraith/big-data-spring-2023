\documentclass{article}
\usepackage[utf8]{inputenc}
\title{The Hadoop Distributed File System}
\author{wbg231 }
\date{December 2022}
\newcommand{\R}{$\mathbb{R}$}
\newcommand{\B}{$\beta$}
\newcommand{\A}{$\alpha$}
\newcommand{\D}{\Delta}

\newcommand{\avector}[2]{(#1_2,\ldots,#1_{#2})}
\newcommand{\makedef}[2]{$\textbf{#1}$:#2 }
\usepackage{tikz,graphicx,hyperref,amsmath,amsfonts,amscd,amssymb,bm,cite,epsfig,epsf,url}

\begin{document}

\maketitle

\section*{the Hadoop Distributed File System}
\begin{itemize}
\item \href{https://brightspace.nyu.edu/d2l/le/lessons/261985/topics/8157424}{paper link}
\section{abstract}
\item Hadoop is defined to store very large data sets reliably and to stream those data sets at high band witch to user applications. 
\item distributing storage and computation across many servers allow resources to grow with demand and remain economical at every size.
\section{intro}
\item hadoop implements map reduce
\item all hadoop code open source
\item HDFS is the file system of hadoop
\item metadata is stored on servers called Name nodes, application data are stored on servers called Data Nodes
\item all servers are fully connected.
\item data is replicated on multiple servers to increase reliability, and also allows for increased data transfer as band with is multiplied. 
\item can partition multiple sections of a file system into different name nodes, in order to add further levels of pluralization
\section{architecture}
\subsection{name node}
\item the HDFS name space is a hierarchy of files and directories. 
\item files are represented in the name node by inodes, which record attributes of the files like permissions 
\item file content are split into file blocks that are replicated by multiple data nodes. 
\item the NameNode main taints the namespace tree and the mapping of file blocks to DataNodes
\item when writing data the name node selects multiple data nodes to replicate and store the data.
\item there is one name node for each cluster that is kept entirely in RAM
\item the inode data and list of blocks belonging to each file comprise the meta data of the name system called the image
\item the persistent record of the image stored in the local hosts native files is called a checkpoint 
\item the name node also stored a record of changes made to the image called the journal 
\item on restart the state of the name space comes from the  
\subsection{data nodes}
\item each block replica on a data node is represented by two files in the local hosts file system. the first file contains the data its self the second is block's metadata.
\item there is a check to make sure all data notes meet the correct software version prior to run 
\item data nodes storage there storage id which uniquely identify them regardless of IP address
\item data node also reports what block replica it has 
\item the data nodes send heartbeats which are reports that they are still running, the name node responds to these heartbeats with further instructions 
\section{HDFS Client}
\item the HDFS client is how user apps interact with the file system 
\item has most typical file structure operations
\item HDFS client directly interacts with the data nodes as well as name nodes

\end{itemize}
\end{document}
